\documentclass[12pt]{IEEEtran}

\author{Ralph Krimmel}
\title{An overview of current BGP security problems}


\begin{document}
	\maketitle
	\begin{abstract}
	\end{abstract}
	\section{Introduction}
	Communication in today's internet is possible by the interplay of many protocols. On the so called internet layer the Internet Protocol \emph{IP} is the primary protocol that is used to forward network packets (datagrams) to the respective target hosts or network. 
	IP networks are grouped as autonomous systems, \emph{AS}, networks under control of one organization. 
	The routing inside or between networks is done by every host that speaks IP, but most importantly by special hosts, called routers. 
	A router has to decide which will be the next hop the datagram has to be moved to. 
	This can either be inside or between autonomous systems. 
	To perform this decision, routers make use of additional protocols which are classified as intradomain routing protocols for routing in, and interdomain routing protocols for routing between ASes. 
	The prevalent interdomain routing protocol is BGP, the \emph{Border Gateway Protocol}. 
	In practise BGP works basically well but the lack of security mechanisms and performance guarantees makes it vulnerable against attacks or configuration mistakes. 
	There have been a few incidents in the past where misconfigured routers reduced the functionality of bigger parts of the internet. 
	With manipulated BGP messages, even more harm could be caused if those messages are intelligently sent to specific and important targets. 
	Therefore it is still highly important to research methods to make BGP secure, performant and reliable. 
	\section{The Border Gateway Protocol}
	The Border gateway protocol is an incremental protocol that enables border routers of adjacent autonomous systems to exchange information on how to reach blocks of IP adresses. 
	Incremental means, that BGP transfers changes of routes and not complete routing tables in every message. Also BGP has a AS Path attribute to which 
	each BGP router adds its AS number, the unique number each AS has, to the beginning of the path. This is done to prevent routing loops.
	

	\subsection{Problems with BGP}
		\subsubsection{Prefix hijacking}
		
       	\subsubsection{Attacks on TCP}
		eavesdropping to learn routing information (not a BGP specific attack)
		MITM (attacks against integrity
			inserting, modifying deleting messages, replay attack
		DoS attack against TCP 
			(sending RST, SYN-Flood, backhoe attack (cutting a link)
		\subsubsection{Attack on the routing policy} 
			using the BGP route attributes
		Exploiting as path length and MED to manipulate route selection by an AS



       \section{BGP security today}
	target:	byzantine robustness
	Currently used: Protection of the TCP connection and defensive filtering
	
%Cryptographig techniques: 
%	Pairwise keying
%	Cryptographic hash functions
%	MACs
%	Diffie-Hellmann%	PKI
%	Certificates	


	\subsection{Protection of a BGP Session between routers}
%	2 goals: Protecting TCP and BGP session itself
%		Proposed solutions:
%			MD5 integrity: Utilize TCP extension that uses a MAC based on MD5
%					-> Protects integrity and prevents replay attacks
%
%			Session and Message Protection:
%				5 Proposed countermeasures:
%					Adding sequence numbers
%					Encryption of all BGP data between peers (shared secret)
%					
%					Adding UPDATE sequence numbers/timestamp
%					New path attribute: PREDECESSOR: identification of last AS before destination
%					digital signatures of all UPDATE fields
%				Disadvantages: BGP needs to be altered
%					       Based on shared secrets => hard to manage
%			Hop Integrity Protocols:
%				Peers can detect modification or replay attacks	
%				Implemented by using sequence numbers, MACs and a PKI to refresh key
%
%			Generalized TTL Security Mechanism
%				Utilizing IP TTL to discard every packet with TTL < MAX-1. 
%				
%				=> Weakly defends against remote attacker, but not against malicious information coming from adjacent peers
%				=> Also, useless in multihop environments
%				
%			IPsec
%				Use IPsec to secure BGP at IP layer 
%				IKE for key management, AH and ESP for packet level security
%				Typically used to secure messages between peers
%				Provides: authenticy, integrity, replay prevention, confidentality, DOS prevention
%			
%
%
	\subsection{Defensive Filtering of suspicious BGP anouncments}
%	Goal: Filter bad and potential malicious announcmentce 
%	Usually ingress and egress filtering based on route policies like:
%			prefixes with special uses
%			bogons/martians (advertisements of adress blocks and AS numbers with no matching allocation data)
%				=> filtering using an updated list of bogons
%			filter out private AS numbers
%			too long AS-Pathes
%			routes to small soubnets (snm > 24) 
%			hard limit of announcments by a neighbour 
%			filtering by customer policies
%				

	\subsection{Routing Registries}
%		Approach: Beeing able to have a global view on correct routes makes it easy to detect attacks
%		This could be achieved by creaint a routing registry that stores the following attributes:
%			prefix ownership
%			Connectivity between ASes
%			routing policies
%			
%		Problems: 
%			Such a registry has to be accurate, complete and secure
%			Routing information sometimes intendet not to be public available
%			Full trust on registry (SPoF)
%			Information may become outdated due to lazy update policies
			
	
	
	%Securing router managment
	%	Protection against physical attacks, SNMP attacks, DoS Attacks
	%       => Basic security, won't mention
			
				
%	All of those described solutions are not sufficient for the protection of BGP

	\section{BGP Security Solutions}
	Multiple complete security architectures have been proposed to 
	\subsection{S-BGP}
%			Validates Path attributes in BGP-Update by utilizing a PKI
%			Data like adress ownership, peer AS identiy, control messages, policy attributes and path vectors can be digitally signed and verified
%			The ownership of a prefix is checked by an out of band mechanism called Adress attestations by the validation of a delegation chain (similar to x509 PKI)
%			Route attestations happen within BGP by appending a new attribute to the BGP UPDATE mesage. Each AS in the AS path signs prior signatures.
%			Problems: 
%				Huge amount of data that needs to be processed and number of possible signers makes this solution computational expensive
				
	\section{Conclusion}				

	
	\section{Literature}

\end{document}
